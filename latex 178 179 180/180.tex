\documentclass{book} 
\usepackage[top=3cm,right=3.5cm,bottom=3cm,left=3cm]{geometry}
\renewcommand{\baselinestretch}{1.7}
\usepackage[utf8]{inputenc}
\usepackage[english]{babel}
\setlength{\parindent}{4em}
\setlength{\parskip}{1em}
\begin{document}
Marttunen and Ahern, Peck, and Laycock recorded a total of 545 and 185 messages respectively, a total that would have been considerably larger if the messages had been subdivided. Third, it is a unit whose parameters are clearly determined by the author of the message—they explicitly choose when to end the message. The major disadvantage of the message as the unit of analysis is that often more than a single idea Of interest is expressed in a single message. We are familiar with email authors who Can never seem to end their messages without "one final point," while Others regularly author extremely sparse messages. Thus, defining the length of a message is as challenging as defining how long a piece of string is. These concerns with grammatically determined units of analysis have promoted some e-researchers to look for units that are defined, not by grammar or syntax, but by meaning.\\
\textbf{The Meaning Unit as Unit of Analysis}  \par
Henri (1991) rejected the process of a priori and authoritatively fixing the size of the unit based on criteria that are not directly related to the construct under study. Instead, she proposes a thematic or meaning unit. Budd, Thorp, and Donohew (1967) define thematic units as "a single thought unit or idea unit that conveys a single item of information extracted from a segment of content" (p. 34). Quoting from Muchielli, Henri (1991) justifies this approach by arguing that "it is absolutely useless to wonder if it is the word, the proposition, the sentence or the paragraph which is the proper unit of meaning, for the unit of meaning is lodged in meaning" (p. 134). The task of explaining what this enigmatic statement meant to pragmatic researchers was taken up by Howell-Richardson and Mellar (1996). Drawing on speechact theory, they explained that transcripts should be viewed with the following question in mind: WI'1at is the purpose of a particular utterance? A change in purpose sets the parameters for the unit. MacDonald (1998) provides a slightly expanded set of guidelines for identifying a speech segment as the unit of analysis. She observes the following:\par 
These authors also evaded some of the difficulties that Henri's scheme presents by sticking to manifest content such as the linguistic properties of the posting and the audience to whom it was directed. Coding a complex, latent construct such as "in-depth processing" with a volatile unit such as Henri's "meaning unit" creates large opportunity for subjective ratings and low reliability. Our discussion Of meaning units thus far illustrates that choosing the unit of analysis for a content analysis of a transcript from an online activity is not an easy task.\par 
Our advice is to try coding using a number of units, checking for ease of identification of the unit, ease of classification of the content, and finally, the reliability of both processes by multiple coders. The value of being systematic may also guide the selection of the unit of analysis, since some variables lend themselves to particular units. For example, if the study is looking at the level of argumentation shown by students, the whole message will most logically reveal this complex variable, whereas a study of frequency of postings may make the time stamp on each message the most logical unit of analysis.
\end{document}