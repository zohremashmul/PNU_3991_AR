\documentclass{book} 
\usepackage[top=3cm,right=3.5cm,bottom=3cm,left=3cm]{geometry}
\renewcommand{\baselinestretch}{1.7}
\usepackage[utf8]{inputenc}
\usepackage[english]{babel}
\setlength{\parindent}{4em}
\setlength{\parskip}{1em}
\begin{document}
Quantitative content analysis usually deals with manifest variables, since they are the only ones to which very high levels of reliability can reasonably be expected among multiple coders. An excellent example ofa large quantitative content analysis is Project H, an international research project that involved 107 researchers in some twenty different countries (Allbritton, 1996). Project H analyzed a large number of mostly manifest variables identified in over 100 Usenet and email list discussions. The complicated processes involved in coordinating such a large research team as well as excellent description of methodological and ethical issues involved in this exemplar content analysis project are provided in a 1996 article by the principal investigators of Project H, Sudweeks and Rafaeli. Their article is appropriately entitled "How do you get a hundred strangers to agree: Computer mediated communication and collaboration.\\
\textbf{THE CODING PROCESS}\par
 Our discussion of quantitative content analysis illustrates the difference between qualitative and quantitative content analysis, yet there are also similarities. Despite our concern with differentiating between different methods that use the same terms, we are cognizant of the research axiom "there can be no quantification without qualification." This expression underlines the necessity of first being able to identifr and categorize the variable before we can count, assign a category, or interpret the content. The challenge is twofold: first we must be able to find a commonly understood and easily distinguishable unit of analysis, and, second, we must be able to reliably and consistently classifr each of these units.\\
\textbf{DEFINING THE UNIT OF ANALYSIS}  \par
The following hypothetical example of an email extract among students in an elearning context provides us with an example of the challenges that confront the e-researcher engaged in quantitative content analysis.
\begin{center}
HI folks,
\end{center}
\par
 i really hate the way we have to answer ALL THESE QUESTIONS before getting the data right—how about you? have very little time this week for this assignment. The last time I tried this I got bogged down with chapter 3 do we really have to know about semiotics, I want to get onto the project analysis first then decide if we need all this theory how about I'll do the first question and you guys do the rest.\\ Your-confused-comrade-in-arms,\\ Terry
\end{document}
